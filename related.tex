\section{Related Work}\label{sec:related}
%There is much related work in the realm of cloud task scheduling.
In the domain of job scheduling in data analytics frameworks like Spark, delay scheduling~\cite{Zaharia2010} has been suggested as the reasonable strategy for
achieving fair sharing for jobs and data locality for tasks.
Some schedulers (like Hadoop YARN's~\cite{Vavilapalli2013} Fair Scheduler) have begun implementing
fixed delay scheduling as a skipcount proportional to the size (in number of nodes) of the compute cluster. 
This approach adapts the interval to the size of the cluster, but not to the usage or load of the 
cluster in real time. Additionally, YARN considers coarse-grained resource allocations for entire frameworks, 
as opposed to the finer-grained tasks we consider.

Sparrow~\cite{Ousterhout2013} is a distributed task scheduler, which focuses on a particular 
set of high-frequency, interactive workloads. Sparrow features distributed, stateless scheduling 
and randomized load balancing of tasks to reduce job latency; however, its completely distributed nature makes enforcing 
fairness policies challenging, and its benefits decrease as workloads stray from the interactive scale.
Our goal, on the other hand, is to provide lower job latency for a diversity of workloads, 
through simple adaptative techniques.

There is also an extensive amount of work on scheduling for custom cluster-management frameworks that attempt to minimize task completion times.
Quincy~\cite{Isard:2009} is a scheduling system which explores the tradeoffs of various types of fairness
policy enforcement, in the presence of data locality concerns. In particular, Quincy focuses on the use
of preemption, to enforce either fairness or locality based on a computed optimal schedule. One caveat to 
this approach is that computing/recalculating a complete schedule causes increased overhead to making
scheduling decisions. Our solution seeks to improve scheduling with very little extra overhead or state, keeping
decisions fast.
Microsoft Apollo~\cite{Boutin2014} is a scheduling framework which uses distributed, 
loosely synchronized schedulers to opportunistically schedule tasks, correcting conflicts should they arise.
Apollo features a service which advertises load to schedulers to facilitate better scheduling decisions. Unlike
our system, Apollo's architecture has tight integration between resource management and job/task scheduling. We
consider systems that schedule between/within jobs, which have to efficiently use resources already acquired and
shared amongst running jobs.

