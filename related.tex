\section{Related Work}\label{sec:related}
%There is much related work in the realm of cloud task scheduling.
Google Borg~\cite{Verma2015} is a cluster manager that features robust fault tolerance and
user isolation using containers. Borg features a centralized scheduler which assigns tasks to machines 
based on fine-grained resource requirements, but Borg does not consider data locality (aside from scheduling
tasks on machines that already have packages/programs that are needed). 
Microsoft Apollo~\cite{Boutin2014} is a scheduling framework which uses distributed, 
loosely synchronized schedulers to opportunistically schedule tasks, correcting conflicts should they arise.
Apollo features a service which advertises load to schedulers to facilitate better scheduling decisions,
but provides only probabilistic fairness guarantees; we
hope to show that simple adaptive feedback metrics are enough to improve centralized scheduling in the presence
of strict fairness enforcement.
Sparrow~\cite{Ousterhout2013} is a distributed task scheduler, which focuses on a particular 
set of high-frequency, interactive workloads. Sparrow features distributed, stateless scheduling 
and randomized load balancing of tasks; however, its completely distributed nature makes enforcing
fairness policies very difficult, and its benefits wane as workloads stray from the interactive scale.
Finally, some schedulers (like Hadoop YARN's~\cite{Vavilapalli2013} Fair Scheduler) have begun implementing
fixed delay scheduling as a skipcount proportional to the size (in number of nodes) of the compute cluster. 
This approach adapts the interval to the size of the cluster, but not to the usage or load of the 
cluster in real time.
